\subsection{Bicubic} \label{subsec:bicubic}

A \textbf{Bicubic} interpolation interpolates from the nearest sixteen source points by performing five cubic interpolations. We use a Lagrange Polynomial on four points to determine a cubic function. That is, given $n+1$ points $(x_0,y_0), (x_1,y_1),\hdots, (x_n, y_n)$, there is a unique degree $n$ polynomial which exactly fits those $n+1$ points

\[P(x)=\sum_{i=0}^ny_i\ell_i(x)\]

where 

\[\ell_i(x)=\prod_{\substack{j=0\\j\neq i}}^n\frac{x-x_j}{x_i-x_j}\]

That $P(x_i)=y_i$ is straightfoward to show: for $j\neq i$, \[\ell_j(x_i)=0\] On the other hand, \begin{align*}\ell_i(x_i)&=\prod_{\substack{j=0\\j\neq i}}^n\frac{x_i-x_j}{x_i-x_j}\\&=1\end{align*} Thus, \begin{align*}P(x_i)&=\sum_{j=0}^ny_j\ell_j(x_i)\\&=\sum_{\substack{j=0\\j\neq i}}^ny_j\ell_j(x_i)+y_i\ell_i(x_i)\\&=0+y_i\cdot1\\&=y_i\end{align*}

To establish uniqueness, we show that if $q(x)$ is another degree-$n$ interpolating polynomial on these $n+1$ points, then $r(x)=p(x)-q(x)=0$. At each $x_i$, we have that $r(x_i)=0$, hence $r$ has $n+1$ roots and can be factored

\[r(x)=a(x-x_0)(x-x_1)\cdots(x-x_n)\]

However, $r$ must be of degree $n$, yet the expansion of the above expression results in a degree-$(n+1)$ polynomial $r$. This forces $a=0$. Thus, $r(x)=0$ and $p(x)=q(x)$. 

To interpolate at point $(x, y)$, we find sixteen nearby source points and interpolate horizontally four times. We then interpolate one final time, vertically. 
\begin{figure}[H]
    \centering
    \includegraphics[scale=0.5]{images/lenna-bicubic.jpg}
    \caption{Bicubic Interpolation with scale $(\sqrt{2}, \sqrt{2})$}
    \label{fig:bicubic}
\end{figure}