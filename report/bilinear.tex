\subsection{Bilinear} \label{subsec:bilinear}

\textbf{Bilinear Interpolation} resizes an image in the same vein as Nearest Neighbor, but applies linear interpolation three times. Specifically, if we have points $(x_0, y_0)$ and $(x_1, y_1)$, we perform linear interpolation at a point $x$ between $x_0$ and $x_1$:

\[y = y_0+(x-x_0)\frac{y_1-y_0}{x_1-x_0}\]

to determine the value of $y$. As a matter of convention, let $y=i(x,x_0, y_0, x_1, y_1)$ denote the linear interpolation at $x$ on points $(x_0, y_0)$ and $(x_1, y_1)$. Our Bilinear Interpolation is obtained with the following transformation

\[I'(x, y) = i\left(\frac{y}{r_y}, Y_0, p_0, Y_1, p_1\right)\]

where

\begin{align*}
    p_0 &= i\left(\frac{x}{r_x}, X_0, I(X_0, Y_0), X_1, I(X_1, Y_0)\right)\\
    p_1 &= i\left(\frac{x}{r_x}, X_0, I(X_0, Y_1), X_1, I(X_1, Y_1)\right)
\end{align*}

and

\begin{align*}
    X_0&=\left\lfloor\frac{x}{r_x}\right\rfloor\\
    X_1&=\left\lceil\frac{x}{r_x}\right\rceil\\
    Y_0&=\left\lfloor\frac{y}{r_y}\right\rfloor\\
    Y_1&=\left\lceil\frac{y}{r_y}\right\rceil\\
\end{align*}

In other words, we interpolate twice on the $x$-values, then a third time on the $y$-value. It is straightforward to show that the result is the same after interpolating twice on the $y$-values, then a third time on the $x$-value.

\begin{figure}[H]
    \centering
    \includegraphics[scale=0.5]{images/lenna-bilinear.jpg}
    \caption{Bilinear Interpolation with scale $(\sqrt{2}, \sqrt{2})$}
    \label{fig:lenna-bilinear}
\end{figure}